\hypertarget{a00004}{}\subsection{example-\/program.\+c}

\begin{DoxyCodeInclude}
\textcolor{comment}{/*}
\textcolor{comment}{ * This file contains an example of how a Contiki program looks.}
\textcolor{comment}{ *}
\textcolor{comment}{ * The program opens a UDP broadcast connection and sends one packet}
\textcolor{comment}{ * every second.}
\textcolor{comment}{ */}

\textcolor{preprocessor}{#include "contiki.h"}
\textcolor{preprocessor}{#include "contiki-net.h"}

\textcolor{comment}{/*}
\textcolor{comment}{ * All Contiki programs must have a process, and we declare it here.}
\textcolor{comment}{ */}
PROCESS(example\_program\_process, \textcolor{stringliteral}{"Example process"});

\textcolor{comment}{/*}
\textcolor{comment}{ * To make the program send a packet once every second, we use an}
\textcolor{comment}{ * event timer (etimer).}
\textcolor{comment}{ */}
\textcolor{keyword}{static} \textcolor{keyword}{struct }etimer timer;

\textcolor{comment}{/*---------------------------------------------------------------------------*/}
\textcolor{comment}{/*}
\textcolor{comment}{ * Here we implement the process. The process is run whenever an event}
\textcolor{comment}{ * occurs, and the parameters "ev" and "data" will we set to the event}
\textcolor{comment}{ * type and any data that may be passed along with the event.}
\textcolor{comment}{ */}
PROCESS\_THREAD(example\_program\_process, ev, data)
\{
  \textcolor{comment}{/*}
\textcolor{comment}{   * Declare the UDP connection. Note that this *MUST* be declared}
\textcolor{comment}{   * static, or otherwise the contents may be destroyed. The reason}
\textcolor{comment}{   * for this is that the process runs as a protothread, and}
\textcolor{comment}{   * protothreads do not support stack variables.}
\textcolor{comment}{   */}
  \textcolor{keyword}{static} \textcolor{keyword}{struct }uip\_udp\_conn *c;
  
  \textcolor{comment}{/*}
\textcolor{comment}{   * A process thread starts with PROCESS\_BEGIN() and ends with}
\textcolor{comment}{   * PROCESS\_END().}
\textcolor{comment}{   */}  
  PROCESS\_BEGIN();

  \textcolor{comment}{/*}
\textcolor{comment}{   * We create the UDP connection to port 4321. We don't want to}
\textcolor{comment}{   * attach any special data to the connection, so we pass it a NULL}
\textcolor{comment}{   * parameter.}
\textcolor{comment}{   */}
  c = udp\_broadcast\_new(UIP\_HTONS(4321), NULL);
  
  \textcolor{comment}{/*}
\textcolor{comment}{   * Loop for ever.}
\textcolor{comment}{   */}
  \textcolor{keywordflow}{while}(1) \{

    \textcolor{comment}{/*}
\textcolor{comment}{     * We set a timer that wakes us up once every second. }
\textcolor{comment}{     */}
    etimer\_set(&timer, CLOCK\_SECOND);
    PROCESS\_WAIT\_EVENT\_UNTIL(etimer\_expired(&timer));

    \textcolor{comment}{/*}
\textcolor{comment}{     * Now, this is a the tricky bit: in order for us to send a UDP}
\textcolor{comment}{     * packet, we must call upon the uIP TCP/IP stack process to call}
\textcolor{comment}{     * us. (uIP works under the Hollywood principle: "Don't call us,}
\textcolor{comment}{     * we'll call you".) We use the function tcpip\_poll\_udp() to tell}
\textcolor{comment}{     * uIP to call us, and then we wait for the uIP event to come.}
\textcolor{comment}{     */}
    tcpip\_poll\_udp(c);
    PROCESS\_WAIT\_EVENT\_UNTIL(ev == tcpip\_event);

    \textcolor{comment}{/*}
\textcolor{comment}{     * We can now send our packet.}
\textcolor{comment}{     */}
    uip\_send(\textcolor{stringliteral}{"Hello"}, 5);

    \textcolor{comment}{/*}
\textcolor{comment}{     * We're done now, so we'll just loop again.}
\textcolor{comment}{     */}
  \}

  \textcolor{comment}{/*}
\textcolor{comment}{   * The process ends here. Even though our program sits is a while(1)}
\textcolor{comment}{   * loop, we must put the PROCESS\_END() at the end of the process, or}
\textcolor{comment}{   * else the program won't compile.}
\textcolor{comment}{   */}
  PROCESS\_END();
\}
\textcolor{comment}{/*---------------------------------------------------------------------------*/}
\end{DoxyCodeInclude}
 