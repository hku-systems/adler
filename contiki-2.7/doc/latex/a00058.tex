\hypertarget{a00058}{}\subsection{The Contiki build system}
\label{a00058}\index{The Contiki build system@{The Contiki build system}}


The Contiki build system is designed to make it easy to compile Contiki applications for either to a hardware platform or into a simulation platform by simply supplying different parameters to the {\ttfamily make} command, without having to edit makefiles or modify the application code.

The file example project in examples/hello-\/world/ shows how the Contiki build system works. The {\ttfamily hello-\/world.\+c} application can be built into a complete Contiki system by running {\ttfamily make} in the examples/hello-\/world/ directory. Running {\ttfamily make} without parameters will build a Contiki system using the {\ttfamily native} target. The {\ttfamily native} target is a special Contiki platform that builds an entire Contiki system as a program that runs on the development system. After compiling the application for the {\ttfamily native} target it is possible to run the Contiki system with the application by running the file {\ttfamily hello-\/world.\+native}. To compile the application and a Contiki system for the E\+SB platform" the command {\ttfamily make T\+A\+R\+G\+ET=esb} is used. This produces a hello-\/world.\+esb file that can be loaded into an E\+SB board.

To compile the hello-\/world application into a stand-\/alone executable that can be loaded into a running Contiki system, the command {\ttfamily make hello-\/world.\+ce} is used. To build an executable file for the E\+SB platform, {\ttfamily make T\+A\+R\+G\+ET=esb hello-\/world.\+ce} is run.

To avoid having to type {\ttfamily T\+A\+R\+G\+ET=} every time {\ttfamily make} is run, it is possible to run {\ttfamily make T\+A\+R\+G\+ET=esb savetarget} to save the selected target as the default target platform for subsequent invocations of {\ttfamily make}. A file called {\ttfamily Makefile.\+target} containing the currently saved target is saved in the project\textquotesingle{}s directory.\hypertarget{a00058_buildsystem-makefiles}{}\subsubsection{Makefiles used in the Contiki build system}\label{a00058_buildsystem-makefiles}
The Contiki build system is composed of a number of Makefiles. These are\+:
\begin{DoxyItemize}
\item {\ttfamily Makefile}\+: the project\textquotesingle{}s makefile, located in the project directory.
\item {\ttfamily Makefile.\+include}\+: the system-\/wide Contiki makefile, located in the root of the Contiki source tree.
\item {\ttfamily Makefile.\$(T\+A\+R\+G\+ET)} (where \$(T\+A\+R\+G\+ET) is the name of the platform that is currently being built)\+: rules for the specific platform, located in the platform\textquotesingle{}s subdirectory in the platform/ directory.
\item {\ttfamily Makefile.\$(C\+PU)} (where \$(C\+PU) is the name of the C\+PU or microcontroller architecture used on the platform for which Contiki is built)\+: rules for the C\+PU architecture, located in the C\+PU architecture\textquotesingle{}s subdirectory in the cpu/ directory.
\item {\ttfamily Makefile.\$(A\+PP)} (where \$(A\+PP) is the name of an application in the apps/ directory)\+: rules for applications in the apps/ directories. Each application has its own makefile.
\end{DoxyItemize}

The Makefile in the project\textquotesingle{}s directory is intentionally simple. It specifies where the Contiki source code resides in the system and includes the system-\/wide Makefile, {\ttfamily Makefile.\+include}. The project\textquotesingle{}s makefile can also define in the {\ttfamily A\+P\+PS} variable a list of applications from the apps/ directory that should be included in the Contiki system. The Makefile used in the hello-\/world example project looks like this\+:


\begin{DoxyCode}
CONTIKI = ../..
all: hello-world
include $(CONTIKI)/Makefile.include
\end{DoxyCode}


First, the location of the Contiki source code tree is given by defining the {\ttfamily C\+O\+N\+T\+I\+KI} variable. Next, the name of the application is defined. Finally, the system-\/wide {\ttfamily Makefile.\+include} is included.

The {\ttfamily Makefile.\+include} contains definitions of the C files of the core Contiki system. {\ttfamily Makefile.\+include} always reside in the root of the Contiki source tree. When {\ttfamily make} is run, {\ttfamily Makefile.\+include} includes the {\ttfamily Makefile.\$(T\+A\+R\+G\+ET)} as well as all makefiles for the applications in the {\ttfamily A\+P\+PS} list (which is specified by the project\textquotesingle{}s {\ttfamily Makefile}).

{\ttfamily Makefile.\$(T\+A\+R\+G\+ET)}, which is located in the platform/\$(T\+A\+R\+G\+ET)/ directory, contains the list of C files that the platform adds to the Contiki system. This list is defined by the {\ttfamily C\+O\+N\+T\+I\+K\+I\+\_\+\+T\+A\+R\+G\+E\+T\+\_\+\+S\+O\+U\+R\+C\+E\+F\+I\+L\+ES} variable. The {\ttfamily Makefile.\$(T\+A\+R\+G\+ET)} also includes the {\ttfamily Makefile.\$(C\+PU)} from the cpu/\$(C\+PU)/ directory.

The {\ttfamily Makefile.\$(C\+PU)} typically contains definitions for the C compiler used for the particular C\+PU. If multiple C compilers are used, the {\ttfamily Makefile.\$(C\+PU)} can either contain a conditional expression that allows different C compilers to be defined, or it can be completely overridden by the platform specific makefile {\ttfamily Makefile.\$(T\+A\+R\+G\+ET)}. 