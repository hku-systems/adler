\hypertarget{a00072}{}\subsection{Running Contiki with u\+I\+Pv6 and S\+I\+C\+Slowpan support on Atmel R\+A\+V\+EN hardware}
\label{a00072}\index{Running Contiki with u\+I\+Pv6 and S\+I\+C\+Slowpan support on Atmel R\+A\+V\+E\+N hardware@{Running Contiki with u\+I\+Pv6 and S\+I\+C\+Slowpan support on Atmel R\+A\+V\+E\+N hardware}}


This tutorial explains how to run Contiki with I\+Pv6 and 6lowpan support on Atmel R\+A\+V\+EN hardware.  


This tutorial explains how to run Contiki with I\+Pv6 and 6lowpan support on Atmel R\+A\+V\+EN hardware. 

\hypertarget{a00072_toc}{}\subsubsection{Table of contents}\label{a00072_toc}
introduction~\newline
 hardware~\newline
 software~\newline
 overview~\newline
 installation~\newline
 running~\newline
 advanced~\newline
 issues~\newline
 annex~\newline




 \hypertarget{a00072_introduction}{}\subsubsection{Introduction}\label{a00072_introduction}
This tutorial explains how to run Contiki with I\+Pv6 and 6lowpan support on Atmel R\+A\+V\+EN evaluation kit (A\+T\+A\+V\+R\+R\+Z\+R\+A\+V\+EN) hardware. We present basic example system architecture and application scenarios, as well as instructions to run more advanced demos.



 \hypertarget{a00072_hardware}{}\subsubsection{Hardware requirements}\label{a00072_hardware}
To run the demo, you will need at least \begin{DoxyItemize}
\item one A\+VR R\+A\+V\+EN board, which embeds an A\+Tmega1284P and an A\+Tmega3290P micro controller (M\+CU) as well as an A\+T86\+R\+F230 802.\+15.\+4 radio chip. \item one RZ U\+SB stick, which embeds an A\+T90\+U\+S\+B1287 M\+CU and an A\+T86\+R\+F230 802.\+15.\+4 radio chip. \item one PC running Windows to program the chips. For the demo itself, a PC running Linux or Windows. \item one On-\/chip programming platform. We recommend Atmel J\+T\+A\+G\+I\+CE mk\+II.\end{DoxyItemize}
\begin{DoxyNote}{Note}
Links to detailed hardware documentation are in \hyperlink{a00072_annex_hardware}{Annex -\/ Atmel products detailed documentation}
\end{DoxyNote}


 \hypertarget{a00072_software}{}\subsubsection{Software requirements}\label{a00072_software}
To install the demo you need\+: \begin{DoxyItemize}
\item Contiki 2.\+3 or later source code, installed in a directory. In the rest of this tutorial we assume the directory is c\+:/contiki \item Cygwin with \char`\"{}make\char`\"{} utility installed. \item A\+VR Studio 4.\+14 or later \item Win\+A\+V\+R20080610 or later \item Windows drivers installed for the J\+T\+A\+G\+I\+CE mk\+II.\end{DoxyItemize}
Instructions to install these tools are in the section \hyperlink{a00072_annex_software}{Software setup details}.~\newline


To run the demo, you need\+: \begin{DoxyItemize}
\item one PC running Linux with kernel 2.\+6.\+24 or later, with support for the following kernel modules\+: I\+Pv6, usbnet, cdc\+\_\+ether, cdc\+\_\+acm, rndis\+\_\+wlan. \item OR one PC running Windows with I\+Pv6 support. If you use Windows XP, you need Service Pack 3 installed.\end{DoxyItemize}
\begin{DoxyNote}{Note}
On windows XP, if ipv6 support is not enabled, enable it by typing in a shell\+: \begin{DoxyVerb}ipv6 install
\end{DoxyVerb}

\end{DoxyNote}


 \hypertarget{a00072_overview}{}\subsubsection{Demo Overview}\label{a00072_overview}
\hypertarget{a00072_overview_architecture}{}\paragraph{Network Architecture}\label{a00072_overview_architecture}
The network comprises\+: \begin{DoxyItemize}
\item a PC acting as an I\+Pv6 router with an 802.\+15.\+4 interface and an Ethernet interface. \item a R\+A\+V\+EN board acting as an I\+Pv6 host.\end{DoxyItemize}
In the basic demo, you can\+: \begin{DoxyItemize}
\item Ping the R\+A\+V\+EN Board from the router \item Ping the router from the R\+A\+V\+EN board, using the R\+A\+V\+EN board menu \item Browse the web server running on the R\+A\+V\+EN board. The server displays the live temperature measured from the board temperature sensor\end{DoxyItemize}
 

 \hypertarget{a00072_installation}{}\subsubsection{Compiling, installing, configuring}\label{a00072_installation}
\hypertarget{a00072_installation_compiling}{}\paragraph{Compiling the binaries for R\+A\+V\+E\+N and R\+Z U\+S\+B stick}\label{a00072_installation_compiling}
The binaries needed are\+: \begin{DoxyItemize}
\item c\+:/contiki/examples/webserver-\/ipv6/webserver6.elf file for the R\+A\+V\+EN board A\+Tmega1284P \item c\+:/contiki/platform/avr-\/ravenlcd/ravenlcd\+\_\+3290.elf file for the R\+A\+V\+EN board A\+Tmega3290P \item c\+:/contiki/examples/ravenusbstick/ravenusbstick.elf file for the RZ U\+SB Stick A\+T90\+U\+S\+B1287\end{DoxyItemize}
To compile each of them, type in Cygwin\+: \begin{DoxyVerb}cd c:/contiki/examples/webserver-ipv6-raven
make
\end{DoxyVerb}
 \begin{DoxyVerb}cd c:/contiki/platform/avr-ravenlcd
make
\end{DoxyVerb}
 \begin{DoxyVerb}cd c:/contiki/examples/ravenusbstick
make
\end{DoxyVerb}
\hypertarget{a00072_installation_hw}{}\paragraph{Installing the hardware}\label{a00072_installation_hw}
To power the R\+A\+V\+EN, put the E\+X\+T/\+B\+AT jumper in B\+AT position. This will enable power on batteries. If you want to power the R\+A\+V\+EN externally, check instructions in \hyperlink{a00072_advanced_externalboard}{Using an external board for power and Debug}.

The RZ U\+SB Stick needs to be plugged in the PC you will run the demo on. If you plan to run the demo on a Windows PC, you will need to install drivers once contiki is loaded on the stick. Until then, you can exit any driver installation popup.\hypertarget{a00072_installation_loading}{}\paragraph{Programming the boards}\label{a00072_installation_loading}
{\bfseries What you need to do}~\newline


\begin{DoxyItemize}
\item On the R\+A\+V\+EN board, program the binaries on both A\+VR A\+Tmega. \item On the RZ U\+SB Stick, load the binary on the A\+T90\+U\+S\+B1287\end{DoxyItemize}
{\bfseries Hardware connections}~\newline
 \begin{DoxyItemize}
\item Connect the J\+T\+AG connectors to the J\+T\+A\+G\+I\+CE as described in the picture below.\end{DoxyItemize}
 \begin{DoxyItemize}
\item Connect the J\+T\+A\+G\+I\+CE mk\+II to a Windows PC through U\+SB. \item To program (load) each A\+VR, you will need to connect the J\+T\+A\+G\+I\+CE J\+T\+AG connector to the J\+T\+AG pins corresponding to the A\+VR you want to program, as shown in the picture below.\end{DoxyItemize}


{\bfseries To load the binary on each A\+VR in Windows}~\newline


\begin{DoxyItemize}
\item Launch A\+VR Studio and exit any popup window. \item Connect the J\+T\+AG pins of the J\+T\+A\+G\+I\+CE into the J\+T\+AG connector of the target processor. \item In A\+VR Studio, click on \char`\"{}\+Tools\char`\"{}-\/$>$\char`\"{}\+Program A\+V\+R\char`\"{}-\/$>$\char`\"{}\+Auto Connect\char`\"{} \item Go to the \char`\"{}\+Main tab\char`\"{} \item In the \char`\"{}\+Programming mode and target settings\char`\"{} list, select J\+T\+AG \item Select the processor type in the \char`\"{}\+Device\char`\"{} list and click \char`\"{}\+Read Signature\char`\"{}. If the Device signature is read properly, it means A\+VR Studio is properly connected to the A\+VR. \item Go to the \char`\"{}\+Program\char`\"{} tab \item In the \char`\"{}\+E\+L\+F Production file format\char`\"{} section, browse to the binary, then click program\end{DoxyItemize}
\begin{DoxyVerb}For webserver6.elf, set the processor to ATmega1284P
For ravenlcd_3290.elf, set the processor to ATmega3290P
For ravenusbstick.elf, set the processor to AT90USB1287\end{DoxyVerb}


Once the RZ U\+SB Stick is programmed, unplug it from the PC. Note this programmed the fuses, E\+E\+P\+R\+OM, and F\+L\+A\+SH all at once.



 \hypertarget{a00072_running}{}\subsubsection{Running the basic demo}\label{a00072_running}
\hypertarget{a00072_running_router}{}\paragraph{Setting up the router}\label{a00072_running_router}
{\bfseries On Linux}~\newline
 Plug the RZ U\+SB Stick in the PC. It should appear as a U\+SB network interface (e.\+g. usb0).

usb0 should automatically get an I\+Pv6 link local address, i.\+e. fe80\+:\+:0012\+:13ff\+:fe14\+:1516/64. Check this is the case by typing \begin{DoxyVerb}ifconfig
\end{DoxyVerb}
 and checking the addresses of interface usb0

If it does not, add it manually\+: \begin{DoxyVerb}ip -6 address add fe80::0012:13ff:fe14:1516/64 scope link dev usb0
\end{DoxyVerb}


Configure the IP addresses on usb0 \begin{DoxyVerb}ip -6 address add aaaa::1/64 dev usb0
\end{DoxyVerb}


Install the radvd deamon and configure it so the usb0 interface advertises the aaaa\+:\+:/64 prefix as on link and usable for address autoconfiguration.

Radvd configuration (usually in /etc/radvd.conf) \begin{DoxyVerb}interface usb0
{
    AdvSendAdvert on;
    AdvLinkMTU 1280;
    AdvCurHopLimit 128;
    AdvReachableTime 360000;
    MinRtrAdvInterval 100;
    MaxRtrAdvInterval 150;
    AdvDefaultLifetime 200;
    prefix AAAA::/64
    {
        AdvOnLink on;
        AdvAutonomous on;
        AdvPreferredLifetime 4294967295; 
        AdvValidLifetime 4294967295; 
    };
};
\end{DoxyVerb}
 \begin{DoxyNote}{Note}
This values have been carefuly chosen to work on platform using a 16bit clock.
\end{DoxyNote}
Restart the radvd daemon. Example command\+: \begin{DoxyVerb}/etc/init.d/radvd restart
\end{DoxyVerb}


If you get a message that radvd won\textquotesingle{}t start as forwarding isn\textquotesingle{}t enabled, you can run this as root\+:

\begin{DoxyVerb}echo 1 > /proc/sys/net/ipv6/conf/all/forwarding
\end{DoxyVerb}


{\bfseries On Windows}~\newline


Plug the RZ U\+SB Stick in the PC. A \char`\"{}new hardware installation\char`\"{} window should pop up. If it does not, go to \char`\"{}\+Control Panel\char`\"{}-\/$>$ \char`\"{}\+Add Hardware\char`\"{}. Choose \char`\"{}\+Install the driver manually\char`\"{}, then select the search path C\+:\textbackslash{}contiki\textbackslash{}cpu\textbackslash{}avr\textbackslash{}dev\textbackslash{}usb\textbackslash{}I\+NF. Finish the installation.

You now need to get the \char`\"{}interface index\char`\"{} of the U\+SB Stick interface (noted \mbox{[}interface index\mbox{]} in the following) and the Ethernet interface (noted \mbox{[}ethernet interface index\mbox{]} in the following).

In a D\+OS or Cygwin shell, type \begin{DoxyVerb}ipv6 if
\end{DoxyVerb}


As an example, the output might look something like this\+:

\begin{DoxyVerb}...
Interface 7: Ethernet
 ...
 link-layer address: 02-12-13-14-15-16
 preferred link-local fe80::12:13ff:fe14:1516, life infinite
 ...
Interface 4: Ethernet: Local Area Connection
 ...
 link-layer address: 00-1e-37-16-5d-83
 preferred link-local fe80::21e:37ff:fe16:5d83, life infinite
 ...
...
\end{DoxyVerb}


Note the link-\/layer address associated with interface 7 is the U\+SB Stick. Hence \mbox{[}interface index\mbox{]} is 7, \mbox{[}ethernet interface index\mbox{]} is 4 and \mbox{[}ethernet link-\/local address\mbox{]} is fe80\+:\+:21e\+:37ff\+:fe16\+:5d83.

Then you need to \begin{DoxyItemize}
\item Set the U\+SB Stick interface as an advertising interface \item Configure a global IP address on the U\+SB Stick interface \item Add a default route through the Ethernet interface \item Set the aaaa\+:\+:/64 prefix as \char`\"{}on link\char`\"{} and published on the U\+SB Stick interface.\end{DoxyItemize}
To do so, type\+: 
\begin{DoxyCode}
ipv6 ifc [\textcolor{keyword}{interface }index] advertises forwards
ipv6 adu [interface index]/aaaa::1
ipv6 rtu ::/0 [ethernet interface index]/[ethernet link-local address] publish
ipv6 rtu aaaa::/64 [interface index] publish
\end{DoxyCode}
\hypertarget{a00072_running_raven}{}\paragraph{Booting the R\+A\+V\+E\+N boards}\label{a00072_running_raven}
Reboot the R\+A\+V\+EN board. The PC sends router advertisements and the R\+A\+V\+EN Board configures an I\+Pv6 global address based on them. The PC global addresses were set above. Communication is ready.\hypertarget{a00072_running_ping1}{}\paragraph{Pinging the R\+A\+V\+E\+N board from the router}\label{a00072_running_ping1}
On Windows (Cygwin shell) or Linux, type \begin{DoxyVerb}ping6 -n 5 aaaa::11:22ff:fe33:4455
\end{DoxyVerb}
 or \begin{DoxyVerb}ping6 -s aaaa::1 aaaa::11:22ff:fe33:4455
\end{DoxyVerb}
 The router is sending 5 echo requests to the R\+A\+V\+EN board. The R\+A\+V\+EN board answers with 5 echo replies.\hypertarget{a00072_running_ping2}{}\paragraph{Pinging the router from the R\+A\+V\+E\+N board}\label{a00072_running_ping2}
To send a ping from the R\+A\+V\+EN to the router you need to use the R\+A\+V\+EN\textquotesingle{}s joystick and L\+CD screen. Initially, the L\+CD screen should print C\+O\+N\+T\+I\+KI -\/ 6\+L\+O\+W\+P\+AN in a loop. You can navigate the L\+CD menu by using the small joystick just below its lower right corner. To \textquotesingle{}ping\textquotesingle{} push the joystick twice to the right. The R\+A\+V\+EN board sends 4 echo requests to the router, which answers by 4 echo replies.~\newline
 For more information about the L\+CD menu, please see lcdraven.\hypertarget{a00072_running_browse}{}\paragraph{Browsing the R\+A\+V\+E\+N board web server}\label{a00072_running_browse}
In a Web browser, point to \href{http://}{\tt http\+://}\mbox{[}aaaa\+:\+:0011\+:22ff\+:fe33\+:4455\mbox{]}. Then click on \textquotesingle{}Sensor Readings\textquotesingle{}. If no temperature is displayed it means that you need to start the temperature update process on the R\+A\+V\+EN. To do so you must use the R\+A\+V\+EN\textquotesingle{}s L\+CD menu and joystick. Starting from the C\+O\+N\+T\+I\+KI -\/ 6\+L\+O\+W\+P\+AN display navigate to T\+E\+MP and then to S\+E\+ND. You can pick either O\+N\+CE or A\+U\+TO, but in any case you always need to reload the webpage to see the latest temperature reading. ~\newline
 For more information about the L\+CD menu, please see lcdraven.



 \hypertarget{a00072_advanced}{}\subsubsection{Advanced use}\label{a00072_advanced}
\hypertarget{a00072_advanced_externalboard}{}\paragraph{Using an external board for power and Debug}\label{a00072_advanced_externalboard}
To power the R\+A\+V\+EN boards externally and enable debug output on R\+S232, you can use the stk500 board together with the raven.

Power\+: \begin{DoxyItemize}
\item Set the \textquotesingle{}E\+X\+T/\+B\+AT\textquotesingle{} jumper on the R\+A\+V\+EN board to E\+XT \item Attach pin 2 on the bottom strip to G\+ND of your S\+T\+K500 \item Attach pin 1 on the bottom strip to V\+TG of your S\+T\+K500 \item Power the S\+T\+K500\end{DoxyItemize}
Debug Connection \begin{DoxyItemize}
\item Attach pin 4 of the leftmost I/O header to pin \textquotesingle{}T\+XD\textquotesingle{} on your S\+T\+K500 \item Connect the S\+T\+K500\textquotesingle{}s \char`\"{}\+R\+S232\+S\+P\+A\+R\+E\char`\"{} port to a R\+S232 port on a PC \item Connect a terminal program (e.\+g. hyper terminal on Windows, minicom on Linux) to the R\+S232 port on the PC at 57600 Baud, with parity 8\+N1, no flow control \item The raven board will output debug messages to the terminal\end{DoxyItemize}
\begin{DoxyNote}{Note}
To enable specific debugging messages, edit the source file you are interested in (e.\+g. core/net/uip-\/nd6-\/io.\+c for Neighbor Discovery messages debug) and set the macro D\+E\+B\+UG to 1. Then recompile the code, load the new binary on the board and restart the R\+A\+V\+EN.
\end{DoxyNote}
The following image shows this connection, with the red and black being V\+CC and G\+ND. The green wire is debug out\+:



\begin{DoxyNote}{Note}
The output to the R\+S232 converts will only be about 3V, but they are expecting a signal swinging up to V\+TG, or by default 5V. You may have to set V\+TG to 3.\+3V and power the Raven from another source, making sure the G\+N\+Ds of both the S\+T\+K500 and your external source are connected together.
\end{DoxyNote}
\hypertarget{a00072_advanced_details}{}\paragraph{Understanding the setup}\label{a00072_advanced_details}
There is no widely available 802.\+15.\+4 and 6lowpan stack for P\+Cs. As a temporary solution and to be able to connect I\+Pv6 hosts such as R\+A\+V\+EN boards to IP networks, we implemented a \char`\"{}bridge\char`\"{} function on the RZ U\+SB Stick. The RZ U\+SB stick bridges 802.\+15.\+4 packets to Ethernet (The Ethernet interface is emulated on the U\+SB port).

As Ethernet frames and addresses are very different from 802.\+15.\+4 ones, a few adjustements are needed on addresses and some neighbor discovery packets. As a consequence, 802.\+15.\+4 M\+AC addresses configured on both the R\+A\+V\+EN boards and the RZ U\+SB stick must have the format\+:~\newline
 \begin{DoxyVerb}x2:xx:xx:ff:fe:xx:xx:xx
\end{DoxyVerb}
 where x can take any hexadecimal value. Read the section below to change the M\+AC address on one device.\hypertarget{a00072_advanced_eeprom}{}\paragraph{Change a device M\+A\+C address}\label{a00072_advanced_eeprom}
You can change the M\+AC address of a R\+A\+V\+EN board or the RZ U\+SB Stick by setting the 8 first bytes of the E\+E\+P\+R\+OM, following the convention above. You can do this three ways.

The first is to set E\+E\+P\+R\+OM bytes directly in an A\+VR Studio project, in Debug mode

\begin{DoxyItemize}
\item compile the binary file for R\+A\+V\+EN, as explained in \hyperlink{a00072_installation}{Compiling, installing, configuring} \item Connect the J\+T\+AG pins of the J\+T\+A\+G\+I\+CE into the J\+T\+AG connector of the target processor. \item IN A\+VR Studio, go to File-\/$>$open, select the binary just created \item The Debug mode should start \item Click on View-\/$>$memory \item select E\+E\+P\+R\+OM in the menu, then just type in the first 8 bytes the target M\+AC address\end{DoxyItemize}
The second is to reprogram the whole E\+E\+P\+R\+OM individually from the Flash and Fuses.

\begin{DoxyItemize}
\item Connect the J\+T\+AG pins of the J\+T\+A\+G\+I\+CE into the J\+T\+AG connector of the target processor. \item In A\+VR Studio, click on \char`\"{}\+Tools\char`\"{}-\/$>$\char`\"{}\+Program A\+V\+R\char`\"{}-\/$>$\char`\"{}\+Auto Connect\char`\"{} \item Go to the \char`\"{}\+Program\char`\"{} tab \item In the \char`\"{}\+E\+E\+P\+R\+O\+M\char`\"{} section, click on \char`\"{}\+Read\char`\"{} and save the E\+E\+P\+R\+OM content in a file (in hex format) \item Edit this file with a text editor, change the value of the first 8 bytes, save \item In the \char`\"{}\+E\+E\+P\+R\+O\+M\char`\"{} section, check the path to the \char`\"{}\+Input Hex file\char`\"{} is the one to the file you just modified and click on \char`\"{}\+Program\char`\"{}.\end{DoxyItemize}
The third is to modify the default value in the code\+:

\begin{DoxyItemize}
\item Edit the file contiki-\/raven-\/main.\+c in the directory platform-\/raven. You will see the M\+AC address set in a line like\+:\end{DoxyItemize}

\begin{DoxyCode}
\textcolor{comment}{/* Put default MAC address in EEPROM */}
uint8\_t mac\_address[8] EEMEM = \{0x02, 0x11, 0x22, 0xff, 0xfe, 0x33, 0x44, 0x55\};
\end{DoxyCode}


\begin{DoxyItemize}
\item Change this value, recompile and reprogram the elf on the board.\end{DoxyItemize}
\hypertarget{a00072_advanced_fuses}{}\paragraph{Setting the fuses manually}\label{a00072_advanced_fuses}
In case you need to reset the fuses on one A\+VR, do the following\+: \begin{DoxyItemize}
\item In A\+VR Studio, click on \char`\"{}\+Tools\char`\"{}-\/$>$\char`\"{}\+Program A\+V\+R\char`\"{}-\/$>$\char`\"{}\+Auto Connect\char`\"{} \item Go to the \char`\"{}\+Fuses\char`\"{} tab \item In the lower part of the window, set the E\+X\+T\+E\+N\+D\+ED, H\+I\+GH, L\+OW fuses to the following values \begin{DoxyVerb}0xFF, 0x99, 0xE2 for the ATmega1284P on the RAVEN board
0xFF, 0x99, 0xE2 for the ATmega3290P on the RAVEN board
0xFB, 0x99, 0xDE for the AT90USB1287 on the USB Stick
\end{DoxyVerb}
 \item In the same tab, Click on \char`\"{}\+Program\char`\"{} \end{DoxyItemize}
\hypertarget{a00072_advanced_capture}{}\paragraph{Observing packets with Atmel Wireless Services or Wireshark}\label{a00072_advanced_capture}
To view packets being sent over the air, you can use Atmel A\+VR Wireless Services in Sniffer Mode, with the RZ U\+SB Stick. You need the software preinstalled on the RZ U\+SB Stick to do this. Packets are sent on channel 24. Links to detailed information about A\+VR Wireless Services is provided with the RZ U\+SB Stick.

See the \hyperlink{a00067}{R\+Z\+R\+A\+V\+EN U\+SB Stick (Jackdaw)} documentation for more details about using Wireshark.\hypertarget{a00072_adavanced_linux}{}\paragraph{Programming Flash, Fuses, E\+E\+P\+R\+O\+M from a Linux machine}\label{a00072_adavanced_linux}
One can use avrdude to load the binaries in Linux.\hypertarget{a00072_advanced_hc01}{}\paragraph{Using H\+C01 Header Compression Scheme}\label{a00072_advanced_hc01}
I\+E\+TF Internet Draft draft-\/hui-\/6lowpan-\/hc-\/01 defines a stateful header compression mechanism (called H\+C01) which will soon deprecate the stateless header compression mechanism (called H\+C1) defined in R\+F\+C4944. H\+C01 is much more powerfull and flexible, in particular it allows compression of some multicast addresses and of all global unicast addresses.

Contiki is compiled by default with H\+C1 support. To use H\+C01 instead, edit platform/xxx/contiki-\/conf.\+h (replace xxx with avr-\/raven, then avr-\/ravenusb.) and replace the line~\newline
 
\begin{DoxyCode}
\textcolor{preprocessor}{#define SICSLOWPAN\_CONF\_COMPRESSION SICSLOWPAN\_CONF\_COMPRESSION\_HC1}
\end{DoxyCode}
 with 
\begin{DoxyCode}
\textcolor{preprocessor}{#define SICSLOWPAN\_CONF\_COMPRESSION SICSLOWPAN\_CONF\_COMPRESSION\_HC01}
\end{DoxyCode}


Recompile and load Contiki for both the R\+A\+V\+EN A\+Tmega1284P and RZ U\+SB Stick.

If you capture packets being sent over the air (on the 802.\+15.\+4 network), you will see that much more packets are compressed than when H\+C1 is used. Overall, packets sent are much smaller.\hypertarget{a00072_advanced_network}{}\paragraph{Building a more complete network}\label{a00072_advanced_network}
You can integrate the R\+A\+V\+EN boards and RZ U\+SB stick to a more complete I\+Pv6 network by connecting the PC which you plug the RZ U\+SB Stick in to any I\+Pv6 network with correct routing configured.

This way, you will be able to reach the R\+A\+V\+EN boards (to read sensor data for example) from anywhere within this I\+Pv6 network, or even any I\+Pv4 network if v4 to v6 translation mechanisms are used between both networks.

You can also have several R\+A\+V\+EN boards in your setup. If you do so, be sure to configure different M\+AC addresses on each board.



 \hypertarget{a00072_issues}{}\paragraph{Known issues}\label{a00072_issues}
{\bfseries RZ U\+SB Stick Link local address not created on Linux}~\newline


When plugging the RZ U\+SB Stick in a Linux PC, it should automatically configure a link local address (fe80\+:\+:0012\+:13ff\+:fe14\+:1516/64 with default M\+AC address). On some Linux distributions, it seems to fail. To check this, in a terminal, type \begin{DoxyVerb}ifconfig
\end{DoxyVerb}
 If the interface usb0 does not have an I\+Pv6 address starting with fe80\+:\+:, add it manually by typing\+: \begin{DoxyVerb}ip -6 address add fe80::0012:13ff:fe14:1516/64 scope link dev usb0
\end{DoxyVerb}


{\bfseries make version issues}~\newline
 You need to use the \char`\"{}make\char`\"{} executable from Win\+A\+VR. There are compilation issues with G\+NU make coming with Cygwin.



 \hypertarget{a00072_annex}{}\subsubsection{Annex}\label{a00072_annex}
\hypertarget{a00072_annex_contikiDocs}{}\paragraph{Annex -\/ Additional Documentation}\label{a00072_annex_contikiDocs}
\begin{DoxyItemize}
\item U\+SB Stick Platform\+: \hyperlink{a00067}{R\+Z\+R\+A\+V\+EN U\+SB Stick (Jackdaw)} \item User interface on Raven\+:lcdraven \item Wireless libraries for Atmel Radio\+: wireless \item M\+AC for Atmel Radio\+: \hyperlink{a00069}{S\+I\+C\+S\+Lo\+W\+M\+AC Implementation} \item I\+Pv6 Implementation\+: \hyperlink{a00075}{u\+IP I\+Pv6 specific features} \item 6lowpan Implementation\+: \hyperlink{a00070}{6\+Lo\+W\+P\+AN implementation}\end{DoxyItemize}
\hypertarget{a00072_annex_hardware}{}\paragraph{Annex -\/ Atmel products detailed documentation}\label{a00072_annex_hardware}
{\bfseries R\+A\+V\+EN evaluation and starter kits}~\newline
 \begin{DoxyItemize}
\item A\+T\+A\+V\+R\+R\+Z\+R\+A\+V\+EN evaluation kit\+: \href{http://www.atmel.com/dyn/products/tools_card.asp?tool_id=4291}{\tt http\+://www.\+atmel.\+com/dyn/products/tools\+\_\+card.\+asp?tool\+\_\+id=4291} \item A\+VR R\+A\+V\+EN board\+: \href{http://www.atmel.com/dyn/products/tools_card.asp?tool_id=4395}{\tt http\+://www.\+atmel.\+com/dyn/products/tools\+\_\+card.\+asp?tool\+\_\+id=4395} \item RZ U\+SB Stick\+: \href{http://www.atmel.com/dyn/products/tools_card.asp?tool_id=4396}{\tt http\+://www.\+atmel.\+com/dyn/products/tools\+\_\+card.\+asp?tool\+\_\+id=4396}\end{DoxyItemize}
{\bfseries R\+A\+V\+EN A\+V\+Rs and Wireless transceivers}~\newline
 \begin{DoxyItemize}
\item A\+Tmega 1284P Mega\+A\+VR\+: \href{http://www.atmel.com/dyn/products/product_card.asp?part_id=4331}{\tt http\+://www.\+atmel.\+com/dyn/products/product\+\_\+card.\+asp?part\+\_\+id=4331} \item A\+Tmega 3290P L\+CD A\+VR\+: \href{http://www.atmel.com/dyn/products/product_card.asp?part_id=4059}{\tt http\+://www.\+atmel.\+com/dyn/products/product\+\_\+card.\+asp?part\+\_\+id=4059} \item A\+T90\+U\+S\+B1287 U\+SB A\+VR\+: \href{http://www.atmel.com/dyn/products/product_card.asp?part_id=3875}{\tt http\+://www.\+atmel.\+com/dyn/products/product\+\_\+card.\+asp?part\+\_\+id=3875} \item A\+T86\+R\+F230 802.\+15.\+4 Transceiver\+: \href{http://www.atmel.com/dyn/products/product_card.asp?part_id=3941}{\tt http\+://www.\+atmel.\+com/dyn/products/product\+\_\+card.\+asp?part\+\_\+id=3941}\end{DoxyItemize}
{\bfseries Additional hardware}~\newline
 \begin{DoxyItemize}
\item A\+T\+S\+T\+K500 evaluation kit \href{http://www.atmel.com/dyn/products/tools_card.asp?tool_id=2735}{\tt http\+://www.\+atmel.\+com/dyn/products/tools\+\_\+card.\+asp?tool\+\_\+id=2735} \item A\+T\+E\+V\+K1100 evaluation kit \href{http://www.atmel.com/dyn/products/tools_card.asp?tool_id=4114}{\tt http\+://www.\+atmel.\+com/dyn/products/tools\+\_\+card.\+asp?tool\+\_\+id=4114} \item A\+VR J\+T\+A\+G\+I\+CE mk\+II debugging platform \href{http://www.atmel.com/dyn/products/tools_card.asp?tool_id=3353}{\tt http\+://www.\+atmel.\+com/dyn/products/tools\+\_\+card.\+asp?tool\+\_\+id=3353}\end{DoxyItemize}
{\bfseries Buying the hardware (part number A\+T\+A\+V\+R\+R\+Z\+R\+A\+V\+EN and A\+T\+J\+T\+A\+G\+I\+C\+E2)}~\newline
 \begin{DoxyItemize}
\item For the U.\+S. you can use \href{http://www.atmel.com/contacts/distributor_check.asp}{\tt http\+://www.\+atmel.\+com/contacts/distributor\+\_\+check.\+asp} \item Digikey \href{http://www.digikey.com/}{\tt http\+://www.\+digikey.\+com/} \item Spoerle \href{http://www.spoerle.com/en/products}{\tt http\+://www.\+spoerle.\+com/en/products} \item Lawicel \href{http://www.lawicel-shop.se}{\tt http\+://www.\+lawicel-\/shop.\+se}\end{DoxyItemize}
\hypertarget{a00072_annex_software}{}\paragraph{Software setup details}\label{a00072_annex_software}
{\bfseries Contiki}~\newline
 Download Contiki code from \href{http://www.sics.se/contiki}{\tt http\+://www.\+sics.\+se/contiki} and extract the source code. We assume the directory you extract to is c\+:/contiki.

{\bfseries Cygwin}~\newline
 \begin{DoxyItemize}
\item Download Cygwin from \href{http://www.cygwin.com}{\tt http\+://www.\+cygwin.\+com} \item Launch the setup executable \item Follow the instructions until you reach the Window \char`\"{}\+Cygwin 
\+Setup -\/ Select Packages\char`\"{} \item In this window, expand the \char`\"{}\+Devel\char`\"{} item and\end{DoxyItemize}
{\bfseries A\+VR Studio}~\newline
 Download and install A\+VR Studio from \href{http://www.atmel.com/dyn/products/tools_card.asp?tool_id=2725}{\tt http\+://www.\+atmel.\+com/dyn/products/tools\+\_\+card.\+asp?tool\+\_\+id=2725}

{\bfseries Win\+A\+VR}~\newline
 Win\+A\+VR which contains a number of A\+VR tools such as the avr-\/gcc compiler.

Download and install Win\+A\+VR latest version from \href{http://winavr.sourceforge.net/}{\tt http\+://winavr.\+sourceforge.\+net/}

{\bfseries J\+T\+A\+G\+I\+CE mk\+II Drivers}~\newline
 Plug the J\+T\+A\+G\+I\+CE mk\+II in a U\+SB port of a windows PC. Follow the indications to install the Windows drivers automatically. 