Contiki is an open source, highly portable, multi-\/tasking operating system for memory-\/efficient networked embedded systems and wireless sensor networks. Contiki is designed for microcontrollers with small amounts of memory. A typical Contiki configuration is 2 kilobytes of R\+AM and 40 kilobytes of R\+OM.

Contiki provides IP communication, both for I\+Pv4 and I\+Pv6. Contiki and its u\+I\+Pv6 stack are I\+Pv6 Ready Phase 1 certified and therefor has the right to use the I\+Pv6 Ready silver logo.

Many key mechanisms and ideas from Contiki have been widely adopted in the industry. The u\+IP embedded IP stack, originally released in 2001, is today used by hundreds of companies in systems such as freighter ships, satellites and oil drilling equipment. Contiki and u\+IP are recognized by the popular nmap network scanning tool. Contiki\textquotesingle{}s protothreads, first released in 2005, have been used in many different embedded systems, ranging from digital TV decoders to wireless vibration sensors.

Contiki introduced the idea of using IP communication in low-\/power sensor networks networks. This subsequently lead to an I\+E\+TF standard and the I\+P\+SO Aliance, an international industry alliance. T\+I\+ME Magazine listed Internet of Things and the I\+P\+SO Alliance as the 30th most important innovation of 2008.

Contiki is developed by a group of developers from industry and academia lead by Adam Dunkels from the Swedish Institute of Computer Science. The Contiki team currently consists of sixteen developers from S\+I\+CS, S\+AP AG, Cisco, Atmel, New\+AE and TU Munich.

Contiki contains two communication stacks\+: \hyperlink{a00074}{u\+IP} and \hyperlink{a00068}{Rime}. u\+IP is a small R\+F\+C-\/compliant T\+C\+P/\+IP stack that makes it possible for Contiki to communicate over the Internet. Rime is a lightweight communication stack designed for low-\/power radios. Rime provides a wide range of communication primitives, from best-\/effort local area broadcast, to reliable multi-\/hop bulk data flooding.

Contiki runs on a variety of platform ranging from embedded microcontrollers such as the M\+S\+P430 and the A\+VR to old homecomputers. Code footprint is on the order of kilobytes and memory usage can be configured to be as low as tens of bytes.

Contiki is written in the C programming language and is freely available as open source under a B\+S\+D-\/style license.\hypertarget{index_contiki-mainpage-tcpip}{}\subsection{T\+C\+P/\+IP}\label{index_contiki-mainpage-tcpip}
Contiki includes the u\+IP T\+C\+P/\+IP stack (\href{http://www.sics.se/~adam/uip/}{\tt http\+://www.\+sics.\+se/$\sim$adam/uip/}) that provides Contiki with T\+C\+P/\+IP networking support. u\+IP provides the protocols T\+CP, U\+DP, IP, and A\+RP.

\begin{DoxySeeAlso}{See also}
\hyperlink{a00074}{The u\+IP T\+C\+P/\+IP stack documentation} 

The Contiki/u\+IP interface 

Protosockets library
\end{DoxySeeAlso}
\hypertarget{index_contiki-mainpage-rime}{}\subsection{Rime}\label{index_contiki-mainpage-rime}
Rime is a lightweight communication stacks designed for low-\/power radios. Rime provides a wide range of communication primitives suitable for implementing communication-\/bound applications or network protocols.

\begin{DoxySeeAlso}{See also}
\hyperlink{a00068}{The Rime Communication Stack}
\end{DoxySeeAlso}
\hypertarget{index_contiki-mainpage-threads}{}\subsection{Multi-\/threading and protothreads}\label{index_contiki-mainpage-threads}
Contiki is based on an event-\/driven kernel but provides support for both multi-\/threading and a lightweight stackless thread-\/like construct called protothreads.

\begin{DoxySeeAlso}{See also}
Contiki processes 

\hyperlink{a00066}{Protothreads} 

Event timers 

Optional multi-\/threading
\end{DoxySeeAlso}
\hypertarget{index_contiki-mainpage-lib}{}\subsection{Libraries}\label{index_contiki-mainpage-lib}
Contiki provides a set of convenience libraries for memory management and linked list operations.

\begin{DoxySeeAlso}{See also}
Simple timer library 

Memory block management 

Linked list library
\end{DoxySeeAlso}
\hypertarget{index_contiki-mainpage-getting-started}{}\subsection{Getting started with Contiki}\label{index_contiki-mainpage-getting-started}
Contiki is designed to run on many different \hyperlink{a00065}{platforms}. It is also possible to compile and build both the Contiki system and Contiki applications on many different development platforms.

See Getting started with Contiki for the E\+SB platform \hypertarget{index_contiki-mainpage-building}{}\subsection{Building the Contiki system and its applications}\label{index_contiki-mainpage-building}
The Contiki build system is designed to make it easy to compile Contiki applications for either to a hardware platform or into a simulation platform by simply supplying different parameters to the {\ttfamily make} command, without having to edit makefiles or modify the application code.

See \hyperlink{a00058}{The Contiki build system} 